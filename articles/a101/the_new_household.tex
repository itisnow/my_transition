\documentclass{article}
%\usepackage{biblatex}
\usepackage[pdftex,bookmarks=true]{hyperref}

\begin{document}
\title{Resource optimization inside families \\\tt{\small{(Article A101)}}\\\tt{\small{}}}%if published as a PDF add git version here
\author{John Adeyanju\\\small{This work is dedicated to the public domain (CC0 licence) and can be edited at}\\ \scriptsize{\url{www.github.com/itisnow/my_transition/blob/master/articles/a101/the_new_household.tex}}}
\date{April 12, 2013}

\maketitle

\begin{abstract}
The idea for this article comes out from the gay marriage debate.
The purpose was to expose floes in the process, the society uses, to digest such issues.
The debate around the issue does not leave any long lasting effect.
It seams the issue is accepted because we are getting used to the idea and not because we understand the arguments.
For example an argument would be that we have laws economically protecting married couples during and after the marriage.
Because we are all equal under laws all couples should have the possibility to get married.
To show that we still do not accept the right view we need a similar situation that would again make feelings prevail over reason.
We show that this similar situation is reasonable by showing that it is economically better than the current way.
We can reignite the debate with the following proposal: What about a marriage among several people.
This formulation is provocative on purpose.
Most people will immediately link it with polygamy and this will make there feelings prevail.
A less provocative formulation could be, a legal contract among several couples.
\end{abstract}

This article wants to be controversial.
The controversy will be used to show how biased we are when thinking about solutions and problems.
It should help people realize that on some subject they do not want to even consider alternative views or solutions.
This may allow them to question their socio-economic believes.
It is clear that to change the world we need to change people minds.
Politician and companies do what the people wants.
We want better lives.
We convinced ourselves that capitalism will take us there.
It's time to realize that this view is obsolete.
When that happens companies and politics will change instantly.
There are two main reasons for that.
Those entities are comprised of people and their existence depends on satisfying the community.

We start from the assumption that it is too difficult to change the "global" economy therefore we should modify "local" human communities.
One of the problems of globalization is that it forces everyone to the same standards.
This standards are defined as "The minimum common denominator".
If in emerging markets they do not have retirement fund we have to follow.
If not, cost will make us uncompetitive and our economy will slow down.
The main question is: "Can we apply some of the TZM ideas on smaller communities" instead of doing it directly on economies.
What will happen if we change how people live together.
This is a sensible subject, people do not like to change too much their lives.
Change is part of life and it will happen independently of our wishes.

\textbf{Proposal:} what could be the benefits to the current socio-economic system if a family would be comprised of several couple (2-3), leaving together in the same household. 

\section{Analysis}
\subsection{Advantages}

1. The biggest investment a family has to make is for housing.
Big houses are generally cheaper than small apartments.
As there are several people working, the family income is bigger.
The loan needed to purchase the house could be smaller.
The cost of the loan is smaller because the associated risks are smaller.
There is more people earning a living the chance the loan will not be repaid is smaller.
The overall efficiency of living together is also better.
For the same reason a skyscraper is better than several small houses.
For example the resources used for the roof depend on the length and width of the building, but not on the height.
This means that the higher it is the less roof we need per unit of volume.
Sharing resources among several people makes there use more efficient. 
For example each couple does not need its own kitchen or bathroom.
You can find more on this concept here \cite{organism_growth}.

2. Work sharing;
It is now clear that the number of jobs is shrinking.
We are so good at creating goods that a small number of people can provide everything the community needs.
One way to reduce the impact of that on societies would be to reduce the working day \cite{4hourday}.
Although our opinion is that this would make the economy stronger the majority of people cannot understand how can less work create better lives.
This feeling coupled with globalization makes it difficult to implement this idea.
If we cannot change our economies we change our communities.
The 2/3 law of living organism applies to families too.
Increasing the amount of couples does not have the same increase on the money they need.
Ideally there could be a rotation among family members.
Some would work some would do other more important and precious things.
Do not forget that the biggest advantage of distributed jobs is the time the individual has to grow as a social being.
Each year there could be a turnover allowing everyone to contribute their best to the economy and society.

3. Raising kids;
Having kids is very expensive in terms of money but also in time spent.
Taking maternity leave can compromise a woman's career.
In modern societies parenting is pushed aside.
A newborn spend a short time with his parents.
Kindergartens and schools are raising our kids.
While this is economically the most efficient way it is not meant to adapt to the individual kid's need.
It might turn out that kids need smaller communities to fully develop their skills and personalities.
Another thing that kids are missing nowadays is a relative.
A brother or sister with whom to play and share life.
If each couple has only one child each of them would still have a "relative" and as a side product we do not worsen overpopulation.
From the view point of parents they would not put "all bets on one horse".
The sharing needed for this communities to be successful has two advantages.
It uses less resources, and second, it teaches them the benefits of sharing.
They will grow up more altruistic.
They will transfer those skills to society.
For the state bigger families mean less social transfers.
Families use better the resources they have, but more important they need less.
Economy of scale at work.

\subsection{Disadvantages}
1. Less personal space;
This is more a cultural issue.
This can be fixed by a change in the design of houses.
We should also believe that peoples want to solve this kinds of practical daily problems.
Another example of a daily practical task is how to get to work in the "most efficient"\footnote{Here efficiency is subjective} way.
We are sometimes complaining about our journey to work, but we are doing our best and  eventually accepting the outcome.

2. What if a person from one couple becomes too attached to someone from the other couples?
This is something that will happen regardless of what we try to do.
More interesting would be to analyse what are the changes of it to happen.
The argument here could be that because we spend a lot of time with the other couples sooner or later the relation among people will change.
This change will at best break some couple thus making the whole proposal void.
To understand how often it can happen we need to compare it with a similar event for which we know and accept the risk.
What about the working place?
We spend there 8 hours every day constantly interacting with others.
During the working week we spend more time with our co-workers than with our family members.
We all accept that risk as part of the game!
Maybe there would even be a positive effect from this interaction.
There would be a chance to directly compare behaviours and learn from that
\footnote{If your guy is sitting on the couch and drinking bear all day long (and the other one is not), you can show him that it does not need to be that way.}.

\section{Conlustion}
After reading this article you should be able to come up with arguments to show that it is economically worst than single couple families.
If you do please add them.
If indeed you cannot please repeat the exercise as long as it is necessary for you to have one.
The best outcome from the publication of this article would be someone coming out with data, showing that this was tried but it is not feasible.
We would finally have a real reason to dismiss this and not just use the argument: "You are crazy".

\newpage
\bibliography{the_new_household}
\bibliographystyle{plain}

\end{document}