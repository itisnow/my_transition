\documentclass{article}
%\usepackage{biblatex}
\usepackage[pdftex,bookmarks=true]{hyperref}

\begin{document}
\title{My Transition Proposition\\\tt{\small{(Article A100)}}\\\tt{\small{}}}%if published as a PDF add git version here
\author{John Adeyanju\\\small{This document is licenced under CC0 and can be edited at}\\ \scriptsize{\url{www.github.com/itisnow/my_transition/blob/master/articles/a100/my_transition_proposition.tex}}}
\date{October 6, 2012}

\maketitle

\begin{abstract}
A Resource Based Economy (RBE) seams to be a good proposal for a $21^{st}$ century society.
We believe a transition has already started, so the goal is to steer it in this direction.
Section 1 and 2 illustrate some problems we face, and how a RBE would mitigate them.
Section 3 is a concrete economic proposal that would allow as to boost employment, innovation, increase free time and the quality of life, without increasing the cost to a society.
Last but most important section 4\footnote{section 3 and 4 became respectively proposals p100 and p101}, proposes a way to collaborate to speed up this transition.
This collaboration is a way for the community to better understand what it wants, therefore being more effective at presenting it and achieving it.
If along the way you start thinking that this is impossible, rather think how can we make it possible.
\end{abstract}

\section{Introduction}
\label{sec:intro}
Listen to the speech of Severn Suzuki \cite{severn} made at the earth summit at age 11, is remarkable.
It seems that at 11 we are completely capable to understand the problems humanity is facing.
This is amazing considering that in 2011 grown ups still dispute climate change.
Was Severn just a smart kid or is this the way young people are?
According to John Hunter \cite{peace_game} this is the way they are.
He invented a game for children aged 5-7 where they can save the world.
In the game children always managed to create a sustainable system where everyone benefits.
If children can do it this means most social problems are easy.
We do not solve them because we just do not accept the solutions, and not because they are unsolvable!
The solutions are everywhere.
We can find them in books as well as in song lyrics at universities as well as in bars.
It is all there, it always was.
We know it is the truth, but it is hard to accept it.
On one side we have youngsters that can think outside the box and solve any problem.
On the other we still have the same problems we had 50 years ago.
When politician try to solve an issue usually they do not come up with an innovative, new way, but they recycle an old solution.
We have to remember two things.
One problems are easy, two we need new and innovative solutions.
A new way to solve problems would be to comprehend the source of an issue and address it there.
Social problems are not solved because we cover them up.
It is time to put some new ideas on paper to inspire other to follow.
Different people need different explanation to understand the same problem.
The important thing to keep in mind is that in the realm of social problems we can have a \underline{perfect solution} if enough people are involved.
We will try to present some problems in a different way and then point to new solutions.
One way to do that is through a thought experiment.
This idea came from the science world.
If it works for physicist trying to understand the universe, why it should not work for us.
First we create an environment with our assumptions on how the "world" works.
Then we make a prediction on what will happen.
The tricky thing is to get our speculation right.
On the other side during our life in the current socio-economic system we got a lot of experience that we just need to use.
Scientist use this approach when exploring what lies beyond our knowledge.
A. Einstein used it when working on his relative theory.
He also said ethics should be tested in the same way science is, and that truth is what stand the test of experience \cite{einstein}.

Everything came together around 5 years ago with the help of The Zeitgeist Movement (TZM \cite{tzm},\cite{tzfs}) and The Venus Project (VP).
The Resource Based Economy (RBE \cite{rbe}) is a simple idea.
After more then ten years of wandering how and why the search was finally over.
The solution is simple we just need to take money out of our assumptions.
It also shows how difficult it is not to be biased when thinking.
The more we know the harder it is to be innovative.
Researchers \cite{gaming} say you need to spend 10,000 hours on something to be really good at it.
This is about 6 hours a day for 5 years.
It equals to the time you may spend to get a degree.
We could say that all the people having more than 23 years have a good knowledge of our socio-economic system at least in certain areas.
How should we used our "degree" in socio-economic matters?
We want to do something good for ourselves.
The only way to real achieve that is to do good for the community.
As an analogy think of it like working in a system where your salary will be randomly transfer to a person in the community.
No one gets the money they worked for.
The instinct tells us: "This would never work!".
If we use our experience and logic, at the end you see, it could work.
For some aspects it would be better.
To realize that, we need to change our socio-economic model.
The "crazy" proposition above has two goals.
One is to remove money as a motivational instrument\cite{motivation}.
This should diminish inequality based on the contribution we are "allowed" to make (think about the status of bankers vs. farmers).
People would tend to do jobs that matters to them.
Second it create a stronger dependence among the people from that community.
If you harm the person contributing for your salary you harm yourself.
Our model throughout this work will be a RBE.
How are we going to get to a RBE?
At this point in time the transition is still not clear.
We will try to use logic and examples to hint toward a way to do it and we invite you to do the same.
Use your experience and come up with examples that will make all the puzzles fall into place.
Everyone living in our current system, knows what is wrong with it, and can help us fixing it.
By inspire other to take the ideas in this article, blend them with their own, we hope for a snow ball effect that will make it come true.

\section{Why}
\label{sec:why}
Some months ago there was a story about a couple that wanted to give birth to their second child in a different state from where they were living.
They had everything arrange except the visa for the first child.
They were desperate, so they decided to put their child in the hand luggage and pass him through the x-rays.
Fortunately the child survived.
There are millions of such stories, where people risk their lives for nothing and many and up tragically.
From our prospective it is completely unreasonable to risk your child life just to get on a plane.
Sadly it is completely normal, because this is the other face of the same coin (our system).

What should be the goal of our $21^{st}$ century society?
No one should be forced to risk his life because across the border things are better.
The quality of life should be the same everywhere!
There are several other reason why we need a transition, but most can be summed up in one word.
INEQUALITY.
Today even economist agree that inequality is a major brake on growth\cite{inequality}.
Let us make a though experiment.
We have a big country that is cut out from the external world.
The people living in it are all in working age.
At the beginning everyone has a job and earns the same amount.
They have the same level of education, motivation and so on.
We have an equal society.
Because we want a dynamic system we need a force that will drive change and innovation.
That force is free market competition, so people have the \underline{free choice to spend their money} on whatever they like.
To make things easier there is no financial sector, no banks, no interests, no money is created.
This assumptions create a simplified version of our current situation.
Based on our experience we need to predict what will happen to this society.
As long as everyone has some money to spend the economy is strong.
The free market (and choice) fuels quality and variety of products.
At the same time lowering their cost.
Those producing inferior products start loosing their jobs.
The money cycle and the economy starts to slow down.
In the end the majority of the money piles up in a few successful companies, there is not enough money around and the economy stops.
How are we going to restart it?
One way is to reset the system by equally redistribute all the money among everyone irrespective of the success gained in the previous cycle.
This means convincing the top 1\% to give us their (our) money.
In the real world we have the state that keeps the inequality in check.
Are states still capable of doing that.
States are totally dependent on the economy.
It does not produce anything, it just collects taxes.
Italians say you cannot spit in the plate you are eating from.
Forcing the big players that comprise the economy to harm themselves is like a tenant forcing his landlord to cut his rent.
Although they would both benefit the landlord has more options open, this makes it harder to convince him.
The life line of the state and people is so tightly linked to that 1\% that the relationship is not fair any more.
In the real world rich give their money back to the society only on one condition, i.e. if we give them more money in return.
Which kind of misses the point.
As the landlord mentioned above they have more options then us.
If we put too much pressure on them the can go abroad.
A very popular way to restart the economy was to inject new money into it.
The problem there seems to be that the economy got back into recession before the state was able to repay its debt back.

The other big issue here is that the society we end up with is like a war zone.
We are constantly fighting each other.
Everything is scarce so it is kill or be killed.
We need a society where the only limit is imagination, otherwise it is not a system worth implementing.
The next best thing is a transition to a new system inspired to a RBE?
One of the key feature of a social system based on work (like ours) is to be able to create abundance of jobs.
The amount of jobs available should always be double the amount of people.
Currently we are about 7 Billion people so we need at least 14 Billion jobs.
In the free market we are usually happy with 5\% unemployment, but here we take into account only the active population.
For some of those jobs people are forced into doing them, again through scarcity.
How are we going to create Billions of jobs in an instant (e.g one day).
Basically everything should be seen as a job.
From hobbies to leisure activities, learning, resting and so forth.
Everything I do makes me a better person, a better worker, this in turn makes a better society.
What could be the force driving such a system?
I always said to myself, I would work for free as an engineer at NASA.
Even better if they gave me an astronaut position, I would pay them.
I would even clean the floors just to be part of the team.
Why?
I always loved science, technology and exploration.
The opportunity to be on the front line and experience directly those things is priceless.
Of course to an outsider things look better than they really are.
Competitions kills all the fan so we should replace it with a new force.
The new force could be the \underline{free choice to spend our time} on anything we like.
What will happen if we use this force in our previous example.
You may argue that time will aggregate only on a few project.
For example the production of food, as it is not exiting, will be overlooked causing enormous loss of lives.
Here we have two possibilities, make those task exciting or automate them.
Automation is exiting by definition.
How can we make interesting a farm with 100 cows, so that it could potentially employ thousands of (young) people?
The answer is simple.
What do young people like?
Gadgets, games.
Jobs should be seen as a way for people to express their personality.
What if we make an i-farm connected to he internet with phone apps that allows you to remotely control your cows?
We could make a game out of it.
A place were people can use their imagination, collaborate and achieve something real.
We are not yet at a point where virtual worlds are better then reality.
According to \cite{gaming} millions of people live inside them.
Today it is unimaginable how many people such a farm could employ.
Many may say "Using so many resources just for a farm is wasteful".
We know from experience that waist is subjective rather than objective.
Think about the entertainment business, we have it to make people happy.
How can then jobs that make people happy be a waste?
Or are we allowed to be happy only if that create a business?
As a side effect those jobs keep the innovation going.
As we were saying above people having fun at work will accept to be paid less.
The economy was created by us to make our lives better.
If it cannot deliver any more its time to create something new!

\paragraph{Registration at the commune}
As all foreigners I had to register at the commune.
When I was waiting, I was thinking: "What a big waste of resources this is?".
Let assume, there were 5 people accepting applications and 5 behind the scene processing them.
The office is open 5 hours every day.
Lets say each person working there is paid 10 euros per hour.
The total cost is $(5+5)*10*5=500$.
Instead of working, I was waiting there for two hours.
There were 50 more people waiting as well.
If they were all processed within a day the total amount of time lost is $50*5=250$.
The equivalent of 2,500 euros.
In total 3,000\footnote{The numbers here and in the entire article are arbitrary, you can pick your own. They should come from your experience. We use numbers, because they are comparable} euros per day per office.

Another point that we want to make is that people working there are not happy and they do not have a filling of social contribution.
When I presented my application I was asked for a visa.
It is funny because my country does not need a visa for the past 5 years.
Even funnier they have a list of visa free countries printed on the wall.
After hers colleague intervention she checked the list and accepted my application.
I think she would rather be somewhere else.
I do not blame here.
It is a repetitive job with little impact to the community.
Every community needs a certain amount of foreigners to function properly.
But it is very difficult to keep an eye on them.
Nowadays we can think of a solution that will save that money and make people happier.
We could use an online application system.
The security is taken care of, because to complete your registration you must go to the police where they check your data.
This is not done for two reasons.
One is cost.
From the commune point of view they spend only 500.
Maybe not enough to pay for a new online system, but the social cost (3,000) is.
The second is employment.
The state income is several times dependent on employment.
Employees pay taxes, do not ask for benefits and buy things.
For people working is life.
This is way asking them to work less is not accepted well.
This is a crucial thing to overcome in the transition.

\paragraph{How scarce and unsuitable jobs handicap our lives.}
The first issue is the selection process we came up with to kind of select the right people for the job.
On paper is fare, but in reality it is not.
Engineers have always less trouble to find a job than artists.
If you are able to express yourself better with numbers than words you have an advantage.
A lot of people preparing for an exam realizes the uselessness of what they are doing.
When we finally pass it we only showed we know how to pass one, but it does not mean we are better for a particular task.
The difficulty of an exam is proportional to the number of candidates it needs to exclude.
What is the social cost of exclusion?
On one side we do not use the whole potential of our society.
On the other we create a parallel society, crime.
What is the cost of crating a missing job?
To answer we need to understand the cost of crime.
We can estimate it with an example.
Lets us assume the cost of a prisoner equals to the cost of renting a room in a middle class area.
That can be 500 euros.
On the other side a medium priced care can cost $20,000$ euros.
If a person gets a sentence of 1 year for a car theft his cost to the society will be $500*12=6,000$.
Not counting the cost of a legal and education system or for that matter the taxes he did not pay.
It will be much cheaper to just give him a car.
Some will argue: "Why should anyone pay for a car?".
And if no one is paying no car will be made in the first place!
The answer is simple.
We automate boring and repetitive parts of car production.
The fun part (innovation and design) could be done by people wanting to have fun.
The reward would be free cars for everyone.
The next hurdle is: "Can we live in a world with 7 Billion cars?".
Today a car is used on average 1 hour per day.
This is $1/24=4$\%.
The sensible thing to do is car sharing.
In order to share we need to change our behaviour.
If we all need a car in the morning to drive us to work and back, it is difficult to share.
This is why we need a different concept of jobs, working hours that would diminish the need for travelling.
We will assume that those changes can reduce the need for a car so that we could have 1 care per 50 people.
The cost of the car in the previous example would be $20000/50=400$ euros per person.
We will fully appreciate the great invention a car is only when we will realise car ownership is pointless.
This is true for our society as a hole.
We think this video illustrate the concept beautifully \cite{c_c}.
We can all agree that the social cost is huge and unacceptable!
This seams to be a simpler solution that meets our needs.

\paragraph{If everything is counted as a job how are we going to measure the added value each person brings to the society.}
What should his salary be?
It is to complicate to measure, therefore trying to do so will cause inequality.
Everyone could be paid the same amount, remember the driving force is not money.
Where are we going to get the money, for those salaries, from?
How much money do we need?
It depends on the cost of life.
The cost of life is equal to the things we need to buy to survive.
The cost of those is in turn equals to the cost of labour.
As already said automation keeps cost down as do people having fun at work.
The same happens by sharing the burden of work among all the population, thus bringing equality.
The added value, therefore money comes out from a better working society.
"Voir la", the transition has started.
In the next section we will make a concrete example of a transition based on the ideas presented above.

\section{My economic proposal}
\label{sec:economic}
%Finally, I wanted to write this section for ages.
As all revolutionary ideas it is very simple.
The important thing to realise is that it took us 50 years to end up in this mess, so it might take 50 years to get out of it.
We envision small changes that will make life better and in the end lead us to a RBE \cite{rbe}.
It could be applied to Greece and if it really does work it could be extended to the whole Europe.

We should include everyone into the economic process (work that must be done to sustain the society - keep it stable).
In an equal society the load on each person should be the same.
To achieve that we need to adapt the working time to the number of jobs available.
In a market economy there is never enough jobs for everyone.
This should be seen as an advantage that liberates people and not used to enslave them.
Lets assume the unemployment is 20\%.
The working week should be cut to 4 days.
Weekly each employee works 32 hours (instead of 40).
We are now missing 1 working day per week.
This means we need (1/5) 20\% more people to cover existing jobs.
We kind of created 20\% more jobs in a blink of an eye.
The salaries for those new employees comes from decreasing current salaries by the same amount (1/5)\footnote{The decrease of all remuneration by the same amount should be seen as an incentive to increase job distribution. The logic should be less money less time, but same life standard. We should have mandatory jobs, we need them to survive and optional activities that makes us advance as a society. The former cannot be evaluated therefore no remuneration is possible.}.
Like wise \underline{the cost of life decreases for the same amount}!
If the unemployment is less than 20\% we could automate the difference.
After some years we repeat the operation again, by cutting 1 day.
This time we focus on automation, reducing further working load on people!

\paragraph{Let us analyse what the consequences could be.}
We start with the growth model.
We are happy if we grow with a pace of about 5\%.
When an economy grows salaries grow too.
In order for the economy to grow there needs to be more money available.
So we earn more because we need to spend more.
Things and services that where once for free must become payable.
That should in turn employ more people, thus creating better quality of life.
What happens in reality?
Growth in salaries makes us uncompetitive.
The cost of labour increases, thus making it unattractive in respect to countries with smaller economies.
The idea is that those jobs move to cheaper countries and by creating cheaper goods and services, fuel growth in both countries.
The catch of a growth model is that everything must grow.
This must happen to jobs too.
A bigger economy needs to create new job and new job sector to replace the jobs lost.
The system should create 5\% (same as growth pace) more jobs each year plus replacing those gone abroad.
That seems not to happen.
We believe one of the reason is that we do not have a growth in innovation.
It is easier to grow by cutting cost than by innovating.
If a company cuts jobs, to show a growth in profit in a long term, that will prevent the economy from growing in the first place.
This happens because economies are focus too much on money and not on social progress.
Because of the lack of innovation the domino of growth starts to fall apart.
What is the aftermath of this growth?
It just made life more expensive which would not have been a problem if we had had enough jobs to go around with, but we do not.
People become relatively poor.
In Italy unemployed people complained because on the benefits they got, it was hard to survive.
They were getting 20,000 euros.
In the same year working in a small country with a smaller economy, you could earned about 70\% of that.
A solution that tackles both problems (cost of life and innovation) should be better than one solution for each of them.
Those two are interlinked.
Without changing cost we cannot increase innovation.
There might be a rule that says: "In order for the economy to grow at 5\% the innovations must grow at 10\%".
That is maybe high but we need the right kind of innovation.
Improving the petrol engine in a car is less effective than adopting a hybrid or a full electric car. 
We believe that if Europe had started selling electric vehicles in 2000 we would not be in a recession now.
Car sales are very important for our economies.
They are a big industry and the induced economy is as big as they are.
The next recession would be further, because it takes longer to replace all technically obsolete cars than just the broken ones.
You also need to consider the fuel saved.
If in 10 years we had replaced 5\% of petrol driven cars we would have a 5\% decrease on fuel consumption and an equal decrease in future fuel demand. Ideally there could be a 10\% decrease\footnote{Supply and demand determine price. Less consumption means an increase in supply. An increase in supply lowers the price. The oil future market is even more important. Knowing that future consumption will fall decrease prices even more.} in fuel price.
Families owning electric car would save money, but also every citizen owning a car will save some money.
Part of those savings would be used to buy new cars, thus keeping the cycle of growth spinning.
In 2012 we are still subsidising petrol cars as much as 3,000 euros.
We do it to keep short term employment, but we are killing it long term.
Innovation is slow because it is expensive.
Not only the researchers need to be paid, we also need to form them.
Faster innovation means increasing cost, thus less profit.

Lets make a new example here.
If we have 10 researchers paid each 5.
What could happen if we have 20 researchers paid each 2.5 euros.
There should be some kind of improvement with the cost staying the same.
In order for the researchers to survive on half of the salary the cost of life must also be halved.
What are the effects on the whole economy.
For simplicity let us assume that this increase does not result in more innovations and we have only one company and only those 20 people in our world.
We have twice as much people buying.
The company expense remains the same i. e. 50 euros, but because they will sell twice as much the article price can be on half.
Everything needs to change for it to stay the same are the right words here.
We are only used to see prices rise, so it is difficult to imagine how they could fall.
Salaries, pensions and the like are align with inflation at least ones per year.
Inflation is hard to understand, because it seems that this adjustment is actually causing it.
If not reverting this process at least stop doing it would help.
Maybe we could decrease the cost of life by growth pace.
Just because some jobs become too expensive it does not mean we do not need them.
We keep all the jobs and we just correct the system by decreasing the cost.
In respect to ruin peoples lives this options is much nicer.
During a crisis it makes more sense to work less and not more.
Currently the recession is caused by week demand.
Less demand should follow less production of goods and services.
Instead of making them cheaper by increasing the output per person (more work for less money).
If the recession increases the unemployment, we should decrease the working time by the same amount.
People may use some of the extra time to make innovations.
Those would fix the crisis.
A crisis can be seen as a part of the economy cycle.
We do two steps forward and then one backward.
Or it can be seen as an anomaly to be corrected.
In both cases distributing work among everyone is the right way forward, because the impact on people lives is more positive.

A Second important aspect missing in our society is the lack of social growth.
If you today say that immigrants are to blame for the lack of jobs you will find the same support as you would have 10 years ago.
The false cause effect mechanism is simple to explain and comprehend.
Understanding that in reality they keep the economy alive takes a lot of effort, thus time.
We need more quality time to understand the complexity of reality.
We work from 9 to 5 because this is our quality time!
So if we allow our 20 researchers to work only half of the week they now have the time to evolve as citizens, and not only professionally.
At first it seems we will not have more innovations, because the total amount of time spent is the same, but the total amount of experience, knowledge, background is not!
Each person involved brings something new.
Plus research is potentially rewording and fun to do.
Happy people are more productive.
We can still assume the increase in innovation is zero.
The real benefit lies in a community more prone to better accept the little innovation we have.
If people got that subsidising electric car was the way forward or that it is better to prevent disease than to treat it, the new invented vaccine would have a greater impact.
Under this conditions politics and politician will have to change.
The People will not accept silly explanations.
They will (have the time to) better comprehend the true causes of the problem, thus better policies and law will be made.
This proposal is probably not the right one.
I hope out there someone will improve on it.
What we need to keep in mind is our goal to have a better and equal society.
To meet this target we need innovation, thus growth in all sectors (not only where there is profit to be made).
To achieve that we must utilize all the people available irrespective of the cost, their knowledge, age, birth place and so on.

A third aspect on why we have to decrease the cost of life is not only practical, it allows us to increase employment, but is more philosophical.
It allows us to transit toward a RBE.
We will be there when the cost of life will be zero.
It might turn out that in the 21 century the most important innovation will be the start of a new socio-economic system.

\paragraph{One additional issue.}
To make this proposal work in real life we might need a tax change\footnote{As long as this proposal is not adopted globally taxes remain a concern}.
Currently companies pay taxes for each employee.
Hiring 10 additional researchers will always increase the cost for the company.
A solution here would be to tax the total amount (per job title) spent on salaries.
This also illustrates why we need more people involved.
Every solution brings new issues.
We need new, different and more people to solve each and everyone of them.
As a RBE suggests, we need a way to allow everyone to participate.
This will be addressed in the next section.

\section{My social proposal}
\label{sec:social}
Let us quickly make a list on what a Recource Based Economy (RBE)\cite{rbe} says on how a better society could be organized.
More precisely what an interpretations could be:
\begin{enumerate}
\item resources must be equally distributed among all people
\item free access to those resources
\item no political system as we know it
\item it is replace by a technical system that keeps the balance
\item this system is maintained by everyone that would like to contribute, but it needs no human input to run
\item people strive to accumulate, share, develop knowledge (to be able to contribute)
\item there are no need for laws, except technical ones
\end{enumerate}
We have 7 points, but we could also say that 2 follows form 1, 3 and 4 could be joined, 7 follows from 3, 5 and 6.
This is an arbitrary list to keep in mind what we will try to focus on during this section.

If we again look at the growth model one of the growing areas is law.
The total number of laws is bigger every year.
Parliaments are quite productive organisms.
Just 20 years ago there were fewer laws needed concerning trade and companies.
If you have a medium company you need a team of experts to help you navigate through.
If you ask two people they will often disagree on what the right interpretation is, thus needing a judge to help them.
For citizens it is even harder as they do not have as many resources to invest.
Even worst the school system does not prepare you on that level.
This is definitely an area that should not grow any more.
Nowadays people are aware of this issue and they are trying to fix it (see P. Howard TED talk \cite{simple_law}).
In the past laws were used to align the community to one value system.
That was working because the speed of life was slower.
Also helping were less and simpler laws.
Suppose that you want to develop a product, you need to know the technology behind it, but you also need to understand the law applying.
If there is an innovation you will need to adapt technically and legislatively.
By removing law we could be much more efficient.
How can we achieve that and still keep the community's values synchronized.
The technical knowledge does that for us.
We need legislation only because we cannot trust people motivated by profit to always us  technology correctly.
What if the laws of physics were enough.
If physics is able to hold together the whole universe then should it be enough to control human society?
Maybe the way to go is to find a social system that needs only those laws?
Even if we do that a problem remains.
We still need to teach those laws to everyone on the planet.
If you want something to work you must spend a lot of time on it.
More time should be spent on learning science\footnote{Anything based on data that can be evaluated through observation and experiment.} and actively participating in its creation.
Unfortunately after most working day we are too exhausted to even pay attention to the news.
Collectively we do not spends enough time on science and socio-economic issues.
Employed people do not have the time.
Unemployed do not because they have to get to the end of the month first.
Rich people do not, their life need no improvement.
Young people are not taken seriously.
Old are not considered fit any more.
For cost efficiency reason we assign the maintenance of the system (government) to an elected group.
Even if they wanted they will not be able to solve all the problems.
They are just too few to comprehend everything.
If we replaced politics by science we would have a simpler, more efficient and effective solution.
Every innovation will impact us twice.
It would solve a technical and social problem.
Even more important by merging them we get more people involved with both field.
For us to be a successful society knowledge will probably need to grow exponentially (grow fast enough).
To achieve that and keep it going the number of people involved will have to increase at the same pace.
This is good, because it is the only job really worth doing.
There cannot be happier people that those free to invest their time in whatever they like, to contribute for a better society.

\paragraph{Why is a technical system useful?}
The advantage is that it can be represented by equations, thus modelled.
The easier it is to test a model the better we comprehend it.
We can create hypothesis and evaluate them.
If they work we implement them, otherwise we change them and repeat the process.
A concrete example of this is the IT business.
Lets say you have an idea for a good application for mobile devices.
You can easily download some software and simulate the target device on any PC.
This means you can cheaply crate an app test it, refine it, and when it is prefect you launch it, and become a successful entrepreneur.
Everyone that has an idea can build it and presented it to the world.
If you would not be able to simulate the target device, it would be just guess work.
The economic, monetary and trade systems can be simulated.
What we are missing is the possibility for everyone to run those simulations.
We can never predict who is going to invent the next OS, social network or search engine.
If we would elect 700 (size of EU parliament) best people and give them the task to invent those things, they would not succeed.
We only got there because the problem was shared among the community.
The proof of that is that Windows, FB and Google where invented by unknown people, at  their first success.
They were not more educated that others.
Their advantage was the capability to see the society from a different view point.
A second advantage of IT is that it is cheap to make new products, which increases the number of people that can be involved.
We need to gradually replace the political system with a technical one, for which a simulation engine is available to the masses.
In this way more people could express their creativity.
This is already happening.
The world wide web is driven by user generated content.
If content generated in a year would be proposals on how to solve the crisis in Europe, it would be solved many times over.
We can collectively run those simulation in our heads, if we agree on the set of social values we want to conserve.
People contribute on topics that will be rewarded by society.
They create entertainment.
If people would have amazing jobs, where you can impact people by improving their quality of life, we would not need entertainment.
An average person can only save the world in virtual reality.
The driving force behind game like  World of Warcraft \cite{gaming} is the collaboration on innovation to achieve an important goal (epic win).
We have that goal, it is a RBE.
We have creativity thus innovation.
We only need a way to collaborate on a global scale.
An example of such a collaboration is Wiki.
The community slowly and incrementally builds the knowledge about its self.
A good Wiki feature is that it allows everyone to read and write it.
The idea here would be to start the transition by slowly changing the way society works.
A Wiki article would be a proposal for change.
The community than turns that proposal into a bill, that could be sent to politician and ask them to pass it as law.
The involvement is counted as a vote.
The greater the number of people involved the better the quality, and the harder it will be to dismiss it as nonsense.
The disadvantage of Wiki is that its content is stored in one place.
It would be better to use a distributed system.
One where every user has all the documents and all the change history on his PC.
We need a version control system.
One such system is Git \cite{git}.
It was used to develop the Linux operating system.
Nowadays a lot of open source code is develop in a distributed fashion.
The RBE idea of democracy is that everyone is at the same time a citizen and a politician.
All the legislation is available not just for viewing but for editing.
All the changes are treated equally.

\paragraph{How Git works.}
We get the latest version from a remote repository.
We can check what the latest changes were.
We make our own.
We push it back to the remote repository, thus sharing our views.
The remote repository is a location, with a constant address that is visible (reachable) by everyone (through GitHub \cite{github} a free code hosting service).
This location allows us to synchronize our local repositories.
If two people know themselves they can exchange content directly between them.
In the last years Git has become very popular especially in the open source community.
Developers fill safe to share their code with the community.
The code quality is better, because there are more people involved.
Bugs are discovered and fixed sooner.
Contributors fill like they are part of something big.
This is the perfect model, everyone benefits.
Like software development moved form centralize proprietary code to distributed open source, the law making process should follow the same change.
This transition is possible if the ownership moves away from the "inventor" to the community.

\paragraph{What should be the document format?}
Every document should hold two kinds of information.
The legislation part and an explanation as to why it should be this way.
Why this proposition is better for people.
How it implements transition.
The legislation part is comprised by articles.
Each of those should be commented, explained so that everyone understands it.
Those two parts are interweaved inside the document.
To present it to politician we should be able to remove all the comments and present them just the law part.
This is a process used in software development and it is called "Literate Programming".
Programmers use a tool called nuweb\cite{nuweb} to do that.
If the document is written in a predefined format it can easily remove all comments.

When a technical system would be in place writing software will be the way to contribute.
The best theory and practice in distributed software is about version control and documentation.
The second one is more theoretical, because usually there is not enough time to make good documentation.
In a RBE it is vital to have excellent documentation.
Everyone not only the programmers must understand the code.
Computer code cannot be simple if what you are trying to describe, with it, is complex.
The only way to make it comprehensible is to write documentation and not software.
This is why the concepts of "Literate Programming" is so important!
Consider a well known equation $E=mc^2$.
It is simple, beautiful, but what will it take, to explain it to an "average housewife"?
Probably several books.
On the other hand when we talk about it we need a short description, but to apply it correctly we must understand the knowledge behind it.
The same applies to the legislation we want to make as an activist community.

In the beginning, to make it appealing to everyone, it should be as simple as possible.
We will use \LaTeX.
\LaTeX~is as simple as opening Notepad and start typing.
To create nice PDFs you need to install software\footnote{You can find step by step tutorial on how to install those. Most of the times you just need to click install.}, but we are interested in good ideas!
Later when this approach will show its benefits we can agree on more advanced techniques.

"Law makers" and "software makers" do the same thing.
So the tool they are using could be the same.
The difference is that a programmer can find and fix a bug in days a politician needs years.

\section{Conclusion}
\label{sec:conclusion}
In order to collaborate we need to agree on the goal we want to achieve.
We believe, we can all agree that people should come first.
After all we created the economy to make our life better and not worst.
We do not like the current system, but most people would not fill comfortable replacing it with a RBE.
We can promote changes in that direction.
When people realize that small changes inspired by RBE made their life better it will be easier to present bigger ideas.
Some rightfully believe we should not help the current system.
We should help it realize that people lives come first.
We should also try not to waist time arguing something is wrong, and instead propose a solution for it.
It does not matter if it is good or not, others will improve on it.
It also makes you part of the solution.
We respect more things, we are part of.

Why things work in our society?
Why we have at least the filling that law and economy will improve our lives.
We think one of the reason is that we want it to work, and we put a lot of effort into it.
Think about money, it is just paper, with almost no value.
Its value comes from globally excepting it and working hard to get it.
If we would globally agree (which by itself is a lot of work) that RBE is the next step  in our social evolution, it will just work.
\newpage
\bibliography{my_transition}
\bibliographystyle{plain}

\end{document}