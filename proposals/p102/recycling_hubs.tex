\documentclass{article}
%\usepackage{biblatex}
\usepackage[pdftex,bookmarks=true]{hyperref}

\begin{document}
\title{Recycling Hubs\\\tt{\small{(Proposal P102)}}\\\tt{\small{}}}%if published as a PDF add git version here
\author{John Adeyanju\\\small{This work is dedicated to the public domain (CC0 licence) and can be edited at}\\ \scriptsize{\url{www.github.com/itisnow/my_transition/blob/master/proposals/p102/recycling_hubs.tex}}}
\date{February 28, 2013}

\maketitle

\begin{abstract}
This is an extension of what some countries (e.g. Belgium) already do for some items.
For example batteries and light bulbs have to be returned back to shops.
As in all cases one thing is to pass a law and another one is to make the system work.
Although it seams shop managers do not take it seriously it still is a good idea.
This proposal aims to extend this behaviour to all packaging waste.
For the purpose of this article packaging waste is an example of the type of waste that could be recycled in this way.
At the moment we present only an explanation on why this is a good idea.
We do not present any suggestion on how to implement it in the current socio-economic system.
The reader is encouraged to do that as this is an issue that will have to be addressed sooner or later.
\end{abstract}

\paragraph{Proposal:}
We have a big distribution system that we should use to do a second job.
In the first it takes good from the warehouses to shops in the second it should do the opposite.
It should take all the leftovers from shops to the recycling facilities.
The same should be true for the people.
We take the goods we need from shops and bring back to them what we cannot use any more.
A solution that can use one system to do two tasks should always be a preferred way of solving a problem.

\paragraph{Analysis of the recycling process in the current economic environment}
More precisely the first part of that process, that is waste collection.
What happens to the packaging waste left at the side of our street.
It is usually picked up by truck and transported to the recycling facilities somewhere outside the city.
Is this efficient?
We heard several time that the expensive part of recycling is pickup and transportation.
Why is this the case?
\begin{enumerate}
\item packaging has a lot of volume but not a lot of weight.
If you look at truck design you can see that it is true.
Usually they all have a mechanism for compressing waste.
The porpoise of it is to decrease the volume, thus increasing the quantity they can collect in a single trip.
\item this specialized trucks are expensive to buy maintain and operate.
In general trucks are an efficient way of moving goods.
If this would not be true we would not have so many trucks on our roads.
In the current economic model this seams to be the most efficient way.
For this to be true they need to transport a lot of weight for long distances.
We already know that the weight condition is not met, but what about the journey length?
First of all they always make half of the journey empty.
Nowadays even rival companies share the same truck if they can minimize the time the truck is empty.
The second issue is fuel consumption.
The average distance between two trash bins is not much greater than 20 meters.
This means that a several ton track needs to constantly accelerate and brake.
I am sure engineers have done their best to minimize fuel waste, but diesel engines do not operate well in those conditions.
\item transporting goods by truck requires one person, the driver.
In waste transport we need three, the driver and two person loading the waste.
Although waste collection is a very important job it is not valued correctly by the society.
Therefore it should involve as less people as possible.
\end{enumerate}
We believe that this proposal tackles all the above issues, making the waste process more efficient.
\paragraph{Let see why?}
If we consider big shopping molls first, they are usually located outside cities.
We access them by car.
We drive there empty and come back full.
If we would take our packaging waste with us we would have a twofold increase in efficiency.
First of all we do not need a whole department of waste collection, second we transport the waste without any extra cost.
This could result in lower waste collection bills or we could use the money to do something else.
If we consider smaller and therefore closer shops, we would need to carry it manually.
As the waste is only a small percentage of the product weight, taking it back to shops should be less difficult than carrying products home.
The next step is to transfer that waste form the shopping molls to existing recycling facilities.
If we would use waste management track we would solve issue three and partially the second one, but not the first.
A better, but still not perfect solution would be to use the same trucks that deliver goods to shops.
Keep in mind that this is relatively clean type of waste.
After those truck deliver their goods they do the return journey empty.
As long as the recycling and production facilities would not be in the same place we would not have perfect efficiency.
Nevertheless resource savings are significant.

\paragraph{Drawbacks?}
Most of the reasons why we should not do this comes form the current socio-economic system.
As this society is based on jobs everything else is subordinate to job conservation.
How much resources we use directly impacts the number of jobs we have.
What will the impact be if instead of having one truck that can do two tasks we have one truck per task?
A track is assembled from several parts and those parts need to be created as well. 
They are manufactured in different locations.
We need infrastructure to bring them together.
To build that infrastructure we need knowledge and therefore an education system.
Basically the whole society is involved.
The bigger the involvement is the greater GDP we have.
As this is vital to our society we tend to overlook the contradictions such a system (modus operandi) brings.
Think about weapon production.
On one side it increases GDP, thus it keeps the society alive, but on the other when those weapons are used they destroy it.
Another example would be a father keeping this family alive through crime.
As long as the crime does not touch his children it is unlikely the wife will complain.
Of course it always has an impact, but it is difficult to see and quantify.
Because it is hard to explain it is hard to estimate.
On top of that we do not have the time to listen.
We are to busy with GDP growth.
The same GDP growth that is destroying our lives.
Fortunately we have a silver bullet solution for that.
It is about reducing the working day as proposed here \cite{4hourday}\cite{distributed_jobs}.
Job loss is an opportunity to revisit and bring up to date our socio-economic system.
\newpage
\bibliography{recycling_hubs}
\bibliographystyle{plain}

\end{document}