\documentclass{article}
%\usepackage{biblatex}
\usepackage[pdftex,bookmarks=true]{hyperref}

\begin{document}
\title{Distributed Jobs\\\tt{\small{(Proposal P100)}}\\\tt{\small{}}}%if published as a PDF add git version here
\author{John Adeyanju\\\small{This document is licenced under CC0 and can be edited at}\\ \footnotesize{\url{www.github.com/itisnow/my_transition/proposals/p100/distributed_jobs.tex}}}
\date{October 6, 2012}

\maketitle

\begin{abstract}
Money is the most important thing in our society.
We use it to make our lives better.
They also say, that time is money.
As we are running out of money we need an alternative.
We will not increase our quality of life by only using money.
Maybe we could use time to do that?
Time is the most valued think we have.
We should think of it as gold under our mattress.
As all revolutionary ideas it is very simple, but what is the best way to "spend" it? 
This article tries to explain a possible way of doing it.
If you think there is a reason why this proposal cannot work, please help us improve it.
If you can think of a problem you can think of a solution.
\end{abstract}

The end goal is to transit to a better socio-economic system.
One such system could be a Resource Based Economy (RBE) \cite{rbe}.
The important thing to realise is that it took us 50 years to end up in this mess, so it might take 50 years to get out of it.
We propose small changes that will make life better and in the end lead us to a RBE.
It could be tried out on Greece.
If it really does work, it could be extended to the whole Europe.
\paragraph{Proposal:}
We should include everyone into the economic process (work that must be done to sustain the society - keep it stable).
In an equal society the load on each person should be the same.
To achieve that we need to adapt the working time to the number of jobs available.
In a market economy there is never enough jobs for everyone.
This should be seen as an advantage that liberates people and not used to enslave them.
Lets assume the unemployment is 20\%.
The working week should be cut to 4 days.
Weekly each employee works 32 hours (instead of 40).
We are now missing 1 working day per week.
This means we need (1/5) 20\% more people to cover existing jobs.
We kind of created 20\% more jobs in a blink of an eye.
The salaries for those new employees comes from decreasing current salaries by the same amount (1/5)\footnote{The decrease of all remuneration by the same amount should be seen as an incentive to increase job distribution. The logic should be less money less time, but same life standard. We should have mandatory jobs, we need them to survive and optional activities that makes us advance as a society. The former cannot be evaluated therefore no remuneration is possible.}.
Like wise \underline{the cost of life decreases for the same amount}!
If the unemployment is less than 20\% we could automate the difference.
After some years we repeat the operation again, by cutting 1 day.
This time we focus on automation, reducing further working load on people!

\paragraph{Let us analyse what the consequences could be.}
We start with the growth model.
We are happy if we grow with a pace of about 5\%.
When an economy grows salaries grow too.
In order for the economy to grow there needs to be more money available.
So we earn more because we need to spend more.
Things and services that where once for free must become payable.
That should in turn employ more people, thus creating better quality of life.
What happens in reality?
Growth in salaries makes us uncompetitive.
The cost of labour increases, thus making it unattractive in respect to countries with smaller economies.
The idea is that those jobs move to cheaper countries and by creating cheaper goods and services, fuel growth in both countries.
The catch of a growth model is that everything must grow.
This must happen to jobs too.
A bigger economy needs to create new job and new job sector to replace the jobs lost.
The system should create 5\% (same as growth pace) more jobs each year plus replacing those gone abroad.
That seems not to happen.
We believe one of the reason is that we do not have a growth in innovation.
It is easier to grow by cutting cost than by innovating.
If a company cuts jobs, to show a growth in profit in a long term, that will prevent the economy from growing in the first place.
This happens because economies are focus too much on money and not on social progress.
Because of the lack of innovation the domino of growth starts to fall apart.
What is the aftermath of this growth?
It just made life more expensive which would not have been a problem if we had had enough jobs to go around with, but we do not.
People become relatively poor.
In Italy unemployed people complained because on the benefits they got, it was hard to survive.
They were getting 20,000 euros.
In the same year working in a small country with a smaller economy, you could earned about 70\% of that.
A solution that tackles both problems (cost of life and innovation) should be better than one solution for each of them.
Those two are interlinked.
Without changing cost we cannot increase innovation.
There might be a rule that says: "In order for the economy to grow at 5\% the innovations must grow at 10\%".
That is maybe high but we need the right kind of innovation.
Improving the petrol engine in a car is less effective than adopting a hybrid or a full electric car. 
We believe that if Europe had started selling electric vehicles in 2000 we would not be in a recession now.
Car sales are very important for our economies.
They are a big industry and the induced economy is as big as they are.
The next recession would be further, because it takes longer to replace all technically obsolete cars than just the broken ones.
You also need to consider the fuel saved.
If in 10 years we had replaced 5\% of petrol driven cars we would have a 5\% decrease on fuel consumption and an equal decrease in future fuel demand. Ideally there could be a 10\% decrease\footnote{Supply and demand determine price. Less consumption means an increase in supply. An increase in supply lowers the price. The oil future market is even more important. Knowing that future consumption will fall decrease prices even more.} in fuel price.
Families owning electric car would save money, but also every citizen owning a car will save some money.
Part of those savings would be used to buy new cars, thus keeping the cycle of growth spinning.
In 2012 we are still subsidising petrol cars as much as 3,000 euros.
We do it to keep short term employment, but we are killing it long term.
Innovation is slow because it is expensive.
Not only the researchers need to be paid, we also need to form them.
Faster innovation means increasing cost, thus less profit.

Lets make a new example here.
If we have 10 researchers paid each 5.
What could happen if we have 20 researchers paid each 2.5 euros.
There should be some kind of improvement with the cost staying the same.
In order for the researchers to survive on half of the salary the cost of life must also be halved.
What are the effects on the whole economy.
For simplicity let us assume that this increase does not result in more innovations and we have only one company and only those 20 people in our world.
We have twice as much people buying.
The company expense remains the same i. e. 50 euros, but because they will sell twice as much the article price can be on half.
Everything needs to change for it to stay the same are the right words here.
We are only used to see prices rise, so it is difficult to imagine how they could fall.
Salaries, pensions and the like are align with inflation at least ones per year.
Inflation is hard to understand, because it seems that this adjustment is actually causing it.
If not reverting this process at least stop doing it would help.
Maybe we could decrease the cost of life by growth pace.
Just because some jobs become too expensive it does not mean we do not need them.
We keep all the jobs and we just correct the system by decreasing the cost.
In respect to ruin peoples lives this options is much nicer.
During a crisis it makes more sense to work less and not more.
Currently the recession is caused by week demand.
Less demand should follow less production of goods and services.
Instead of making them cheaper by increasing the output per person (more work for less money).
If the recession increases the unemployment, we should decrease the working time by the same amount.
People may use some of the extra time to make innovations.
Those would fix the crisis.
A crisis can be seen as a part of the economy cycle.
We do two steps forward and then one backward.
Or it can be seen as an anomaly to be corrected.
In both cases distributing work among everyone is the right way forward, because the impact on people lives is more positive.

A Second important aspect missing in our society is the lack of social growth.
If you today say that immigrants are to blame for the lack of jobs you will find the same support as you would have 10 years ago.
The false cause effect mechanism is simple to explain and comprehend.
Understanding that in reality they keep the economy alive takes a lot of effort, thus time.
We need more quality time to understand the complexity of reality.
We work from 9 to 5 because this is our quality time!
So if we allow our 20 researchers to work only half of the week they now have the time to evolve as citizens, and not only professionally.
At first it seems we will not have more innovations, because the total amount of time spent is the same, but the total amount of experience, knowledge, background is not!
Each person involved brings something new.
Plus research is potentially rewording and fun to do.
Happy people are more productive.
We can still assume the increase in innovation is zero.
The real benefit lies in a community more prone to better accept the little innovation we have.
If people got that subsidising electric car was the way forward or that it is better to prevent disease than to treat it, the new invented vaccine would have a greater impact.
Under this conditions politics and politician will have to change.
The People will not accept silly explanations.
They will (have the time to) better comprehend the true causes of the problem, thus better policies and law will be made.
This proposal is probably not the right one.
I hope out there someone will improve on it.
What we need to keep in mind is our goal to have a better and equal society.
To meet this target we need innovation, thus growth in all sectors (not only where there is profit to be made).
To achieve that we must utilize all the people available irrespective of the cost, their knowledge, age, birth place and so on.

A third aspect on why we have to decrease the cost of life is not only practical, it allows us to increase employment, but is more philosophical.
It allows us to transit toward a RBE \cite{rbe}.
We will be there when the cost of life will be zero.
It might turn out that in the 21 century the most important innovation will be the start of a new socio-economic system.

\paragraph{One additional issue.}
To make this proposal work in real life we might need a tax change\footnote{As long as this proposal is not adopted globally taxes remain a concern}.
Currently companies pay taxes for each employee.
Hiring 10 additional researchers will always increase the cost for the company.
A solution here would be to tax the total amount (per job title) spent on salaries.
This also illustrates why we need more people involved.
Every solution brings new issues.
We need new, different and more people to solve each and everyone of them.
As a RBE suggests, we need a way to allow everyone to participate.
This will be addressed in the next section.
\newpage
\bibliography{distributed_jobs}
\bibliographystyle{plain}

\end{document}