\documentclass{article}
%\usepackage{biblatex}
\usepackage[pdftex,bookmarks=true]{hyperref}

\begin{document}
\title{The Git Parliament\\\tt{\small{(Proposal P101)}}\\\tt{\small{}}}%if published as a PDF add git version here
\author{John Adeyanju\\\small{This document is licenced under CC0 and can be edited at}\\ \scriptsize{\url{www.github.com/itisnow/my_transition/blob/master/proposals/p101/the_git_parliament.tex}}}
\date{October 6, 2012}

\maketitle

\begin{abstract}
This proposal tries to define a framework in witch the transition can be discussed.
Before discussing a topic we need to agree what the topic is.
How are we going to interact during the discussion, and when an agreement is reached.
This document addresses some of those issues.
\end{abstract}

Let us quickly make a list on what a Recource Based Economy (RBE)\cite{rbe}\cite{tzfs} says on how a better society could be organized.
More precisely what an interpretations could be:
\begin{enumerate}
\item resources must be equally distributed among all people
\item free access to those resources
\item no political system as we know it
\item it is replace by a technical system that keeps the balance
\item this system is maintained by everyone that would like to contribute, but it needs no human input to run
\item people strive to accumulate, share, develop knowledge (to be able to contribute)
\item there are no need for laws, except technical ones
\end{enumerate}
We have 7 points, but we could also say that 2 follows form 1, 3 and 4 could be joined, 7 follows from 3, 5 and 6.
This is an arbitrary list to keep in mind what we will try to focus on during this proposal.

\paragraph{Proposal:}
The involvement of citizens in politics is not only through voting but also with writing the actual laws.
To allow everyone to participate in creating and editing documents we will use a distributed version control system.
The open source software community uses Git.
If it is good enough for them, there no reason why, the open source "law making process" community should not do the same.

\paragraph{What can be gained.}
If we again look at the growth model one of the growing areas is law.
The total number of laws is bigger every year.
Parliaments are quite productive organisms.
Just 20 years ago there were fewer laws needed concerning trade and companies.
If you have a medium company you need a team of experts to help you navigate through.
If you ask two people they will often disagree on what the right interpretation is, thus needing a judge to help them.
For citizens it is even harder as they do not have as many resources to invest.
Even worst the school system does not prepare you on that level.
This is definitely an area that should not grow any more.
Nowadays people are aware of this issue and they are trying to fix it ( see P. Howard TED talk \cite{simple_law}).
In the past laws were used to align the community to one value system.
That was working because the speed of life was slower.
Also helping were less and simpler laws.
Suppose that you want to develop a product, you need to know the technology behind it, but you also need to understand the law applying.
If there is an innovation you will need to adapt technically and legislatively.
By removing law we could be much more efficient.
How can we achieve that and still keep the community's values synchronized.
The technical knowledge does that for us.
We need legislation only because we cannot trust people motivated by profit to always us  technology correctly.
What if the laws of physics were enough.
If physics is able to hold together the whole universe then should it be enough to control human society?
Maybe the way to go is to find a social system that needs only those laws?
Even if we do that a problem remains.
We still need to teach those laws to everyone on the planet.
If you want something to work you must spend a lot of time on it.
More time should be spent on learning science\footnote{Anything based on data that can be evaluated through observation and experiment.} and actively participating in its creation.
Unfortunately after most working day we are too exhausted to even pay attention to the news.
Collectively we do not spends enough time on science and socio-economic issues.
Employed people do not have the time.
Unemployed do not because they have to get to the end of the month first.
Rich people do not, their life need no improvement.
Young people are not taken seriously.
Old are not considered fit any more.
For cost efficiency reason we assign the maintenance of the system (government) to an elected group.
Even if they wanted they will not be able to solve all the problems.
They are just too few to comprehend everything.
If we replaced politics by science we would have a simpler, more efficient and effective solution.
Every innovation will impact us twice.
It would solve a technical and social problem.
Even more important by merging them we get more people involved with both field.
For us to be a successful society knowledge will probably need to grow exponentially (grow fast enough).
To achieve that and keep it going the number of people involved will have to increase at the same pace.
This is good, because it is the only job really worth doing.
There cannot be happier people that those free to invest their time in whatever they like, to contribute for a better society.

\paragraph{Why is a technical system useful?}
The advantage is that it can be represented by equations, thus modelled.
The easier it is to test a model the better we comprehend it.
We can create hypothesis and evaluate them.
If they work we implement them, otherwise we change them and repeat the process.
A concrete example of this is the IT business.
Lets say you have an idea for a good application for mobile devices.
You can easily download some software and simulate the target device on any PC.
This means you can cheaply crate an app test it, refine it, and when it is prefect you launch it, and become a successful entrepreneur.
Everyone that has an idea can build it and presented it to the world.
If you would not be able to simulate the target device, it would be just guess work.
The economic, monetary and trade systems can be simulated.
What we are missing is the possibility for everyone to run those simulations.
We can never predict who is going to invent the next OS, social network or search engine.
If we would elect 700 (size of EU parliament) best people and give them the task to invent those things, they would not succeed.
We only got there because the problem was shared among the community.
The proof of that is that Windows, FB and Google where invented by unknown people, at  their first success.
They were not more educated that others.
Their advantage was the capability to see the society from a different view point.
A second advantage of IT is that it is cheap to make new products, which increases the number of people that can be involved.
We need to gradually replace the political system with a technical one, for which a simulation engine is available to the masses.
In this way more people could express their creativity.
This is already happening.
The world wide web is driven by user generated content.
If content generated in a year would be proposals on how to solve the crisis in Europe, it would be solved many times over.
We can collectively run those simulation in our heads, if we agree on the set of social values we want to conserve.
People contribute on topics that will be rewarded by society.
They create entertainment.
If people would have amazing jobs, where you can impact people by improving their quality of life, we would not need entertainment.
An average person can only save the world in virtual reality.
The driving force behind game like  World of Warcraft \cite{gaming} is the collaboration on innovation to achieve an important goal (epic win).
We have that goal, it is a RBE.
We have creativity thus innovation.
We only need a way to collaborate on a global scale.
An example of such a collaboration is Wiki.
The community slowly and incrementally builds the knowledge about its self.
A good Wiki feature is that it allows everyone to read and write it.
The idea here would be to start the transition by slowly changing the way society works.
A Wiki article would be a proposal for change.
The community than turns that proposal into a bill, that could be sent to politician and ask them to pass it as law.
The involvement is counted as a vote.
The greater the number of people involved the better the quality, and the harder it will be to dismiss it as nonsense.
The disadvantage of Wiki is that its content is stored in one place.
It would be better to use a distributed system.
One where every user has all the documents and all the change history on his PC.
We need a version control system.
One such system is Git \cite{git}.
It was used to develop the Linux operating system.
Nowadays a lot of open source code is develop in a distributed fashion.
The RBE idea of democracy is that everyone is at the same time a citizen and a politician.
All the legislation is available not just for viewing but for editing.
All the changes are treated equally.

\paragraph{How Git works.}
We get the latest version from a remote repository.
We can check what the latest changes were.
We make our own.
We push it back to the remote repository, thus sharing our views.
The remote repository is a location, with a constant address that is visible (reachable) by everyone (through GitHub \cite{github} a free code hosting service).
This location allows us to synchronize our local repositories.
If two people know themselves they can exchange content directly between them.
In the last years Git has become very popular especially in the open source community.
Developers fill safe to share their code with the community.
The code quality is better, because there are more people involved.
Bugs are discovered and fixed sooner.
Contributors fill like they are part of something big.
This is the perfect model, everyone benefits.
Like software development moved form centralize proprietary code to distributed open source, the law making process should follow the same change.
This transition is possible if the ownership moves away from the "inventor" to the community.

\paragraph{What should be the document format?}
Every document should hold two kinds of information.
The legislation part and an explanation as to why it should be this way.
Why this proposition is better for people.
How it implements transition.
The legislation part is comprised by articles.
Each of those should be commented, explained so that everyone understands it.
Those two parts are interweaved inside the document.
To present it to politician we should be able to remove all the comments and present them just the law part.
This is a process used in software development and it is called "Literate Programming".
Programmers use a tool called nuweb\cite{nuweb} to do that.
If the document is written in a predefined format it can easily remove all comments.

When a technical system would be in place writing software will be the way to contribute.
The best theory and practice in distributed software is about version control and documentation.
The second one is more theoretical, because usually there is not enough time to make good documentation.
In a RBE it is vital to have excellent documentation.
Everyone not only the programmers must understand the code.
Computer code cannot be simple if what you are trying to describe, with it, is complex.
The only way to make it comprehensible is to write documentation and not software.
This is why the concepts of "Literate Programming" is so important!
Consider a well known equation $E=mc^2$.
It is simple, beautiful, but what will it take, to explain it to an "average housewife"?
Probably several books.
On the other hand when we talk about it we need a short description, but to apply it correctly we must understand the knowledge behind it.
The same applies to the legislation we want to make as an activist community.

In the beginning, to make it appealing to everyone, it should be as simple as possible.
We will use \LaTeX.
\LaTeX~is as simple as opening Notepad and start typing.
To create nice PDFs you need to install software\footnote{You can find step by step tutorial on how to install those. Most of the times you just need to click install.}, but we are interested in good ideas!
Later when this approach will show its benefits we can agree on more advanced techniques.

"Law makers" and "software makers" do the same thing.
So the tool they are using could be the same.
The difference is that a programmer can find and fix a bug in days a politician needs years.
\newpage
\bibliography{the_git_parliament}
\bibliographystyle{plain}

\end{document}